\documentclass[twocolumn,prl,preprintnumbers,amsmath,amssymb,floatfix]{revtex4}
\usepackage[linktocpage,bookmarksopen,bookmarksnumbered]{hyperref}
\usepackage{graphicx}
\usepackage{dcolumn}

\usepackage{amsmath,graphics,epsfig,color,verbatim,ulem}
\newcommand{\eps}{\epsilon} \renewcommand{\a}{\alpha}
\renewcommand{\b}{\beta} \newcommand{\vR}{{\mathbf{R}}}
\renewcommand{\vr}{{\mathbf{r}}} \newcommand{\vk}{{\mathbf{k}}}
\newcommand{\vK}{{\mathbf{K}}} \newcommand{\vq}{{\mathbf{q}}}
\newcommand{\vQ}{{\mathbf{Q}}} \newcommand{\vPhi}{{\mathbf{\Phi}}}
\newcommand{\vS}{{\mathbf{S}}} \newcommand{\cG}{{\cal G}}
\newcommand{\cF}{{\cal F}} \newcommand{\cD}{{\cal D}}
\newcommand{\Tr}{\mathrm{Tr}} \newcommand{\npsi}{\underline{\psi}}
\newcommand{\vA}{{\mathbf{A}}} \newcommand{\vE}{{\mathbf{E}}}
\newcommand{\vj}{{\mathbf{j}}} \newcommand{\vv}{{\mathbf{v}}}
\newcommand{\kb}{k_B} \newcommand{\cellvol}{}
\newcommand{\trace}{\mbox{Tr}} \newcommand{\ra}{\rangle }
\newcommand{\la}{\langle } \newcommand{\om}{\omega}
\renewcommand{\Im}{\mathrm{Im}} \newcommand{\up}{\uparrow}
\newcommand{\down}{\downarrow}
\newcommand{\nphi}{\underline{\phi}}
\newcommand{\tIm}{\overline{\Im}}


%\setlength{\oddsidemargin}{-0.3in}
%\setlength{\evensidemargin}{-0.30in}
%\setlength{\textwidth}{7.3in}
%\setlength{\topmargin}{-0.1in}
%%%%\setlength{\leftmargin}{0.1in}
%\setlength{\headheight}{0.1in}
%\setlength{\headsep}{0in}
%\setlength{\textheight}{9.8in}

\begin{document}
\special{papersize=8.5in,11in}
\setlength{\pdfpageheight}{\paperheight}
\setlength{\pdfpagewidth}{\paperwidth}

\title{Pade approximation}
\begin{abstract}
\end{abstract}
\maketitle

\begin{widetext}

We represent the Green's function with the following continuous
fraction expansion
\begin{eqnarray}
G(z) = \frac{a_1}{1+\frac{a_2(z-z_1)}{1+\frac{a_3(z-z_2)}{1+\cdots\frac{a_{n-1}(z-z_{n-2})}{1+a_n(z-z_{n-1})}}}}  
\end{eqnarray}
For large $n$, this representation becomes exact.

To build this continuous fraction, we first write $G(z)$ in the
rational form
\begin{eqnarray}
G(z) = \frac{A_n(z)}{B_n(z)}  
\end{eqnarray}
and we compute polynomials $A_n$ and $B_n$ with the following recursion
relation
\begin{eqnarray}
A_{i+1}(z) = A_i(z) + (z-z_i) a_{i+1}A_{i-1}(z)\\
B_{i+1}(z) = B_i(z) + (z-z_i) a_{i+1}B_{i-1}(z)
\end{eqnarray}
and starting conditions
\begin{eqnarray}
A_0=0\\
A_1=a_1\\
B_0=1\\
B_1=1
\end{eqnarray}


We will check the few lowest orders of this continuous
fraction/recursion. At the lowest order, we have
\begin{eqnarray}
\frac{A_1}{B_1} = a_1 
\end{eqnarray}
We use the recursion relation to get $A_2$ and $B_2$:
\begin{eqnarray}
A_2 &=& a_1\\
B_2 &=& 1+(z-z_1)a_2
\end{eqnarray}
which gives
\begin{equation}
\frac{A_2}{B_2}=\frac{a_1}{1+a_2(z-z_1)}  
\end{equation}
In the next order, we get
\begin{eqnarray}
A_3 &=& a_1 + (z-z_2) a_3 a_1\\
B_3 &=& 1+ (z-z_1) a_2 + (z-z_2)a_3
\end{eqnarray}
which gives
\begin{eqnarray}
\frac{A_3}{B_3}=\frac{a_1(1+(z-z_2)a_3)}{1+(z-z_2)a_3+(z-z_1)a_2} =\frac{a_1}{1+\frac{a_2(z-z_1)}{1+a_3(z-z_2)}} 
\end{eqnarray}
In the next order, we have
\begin{eqnarray}
A_4 &=& a_1(1+a_3(z-z_2)) + (z-z_3)a_4 a_1 = a_1(1+a_3(z-z_2) + a_4(z-z_3)) \nonumber\\
B_4 &=& 1+(z-z_1)a_2+(z-z_2)a_3+(z-z_3)a_4(1+(z-z_1)a_2) =
1+a_3(z-z_2)+a_4(z-z_3)+ a_2(z-z_1) (1+a_4 (z-z_3))\nonumber
\end{eqnarray}
which gives
\begin{eqnarray}
\frac{A_4}{B_4} = \frac{a_1(1+a_3(z-z_2) + a_4(z-z_3))}{1+a_3(z-z_2)+a_4(z-z_3)+ a_2(z-z_1) (1+a_4 (z-z_3))}=\\
\frac{a_1}{1+ \frac{a_2(z-z_1) (1+a_4 (z-z_3))}{1+a_3(z-z_2) + a_4(z-z_3)}}=
\frac{a_1}{1+ \frac{a_2(z-z_1)}{1+ \frac{a_3(z-z_2)}{1+a_4 (z-z_3)}}}
\end{eqnarray}
Clearly, the recursion relation can be used to get $A_n$ and $B_n$ for
an arbitrary order $n$.

To represent $G$ with the continuous fraction expansion, we need to
compute coefficients $a_i$ from the value of G at some set of points 
$z_i$ in the complex plane (such as the Matsubara points).

We first notice that
\begin{eqnarray}
G(z_1) &=& a_1\\
G(z_2) &=& \frac{a_1}{1+a_2(z_2-z_1)}\\
G(z_3) &=& \frac{a_1}{1+\frac{a_2(z_3-z_1)}{1+a_3(z_3-z_2)}} 
\end{eqnarray}
and in general $G(z_m) = \frac{A_m(z_m)}{B_m(z_m)}$

We can use the first equation to compute $a_1$, the second to compute
$a_2$, etc. At order $m$ we can get all $a_m$. There exists a
recursion relation to compute all coefficients very efficiently. We
define a matrix $P(i,j)$, which has the following properties
\begin{eqnarray}
P(1,i) &\equiv& G(z_i)\\
P(i,j) &=& \frac{P(i-1,i-1)-P(i-1,j)}{(z_j-z_{i-1})P(i-1,j)}
\end{eqnarray}
We will next show that $P(i,i)$ is
\begin{eqnarray}
P(i,i)  = a_i
\end{eqnarray}

We start with the first order $P(1,1) = a_1$. In the second order we have
\begin{eqnarray}
P(1,2) = G(z_2) = \frac{a_1}{1+a_2(z_2-z_1)}
\end{eqnarray}
hence
\begin{eqnarray}
a_2 = \frac{a_1-G(z_2)}{(z_2-z_1)G(z_2)} = \frac{P(1,1)-P(1,2)}{(z_2-z_1)P(1,2)}
\end{eqnarray}
which is clearly compatible with the above recursion relation.

Next, we compute $P(2,3)$ and $P(3,3)$ with the recursion relation,
and we will check that $P(3,3)=a_3$. We have
\begin{equation}
P(2,3) = \frac{P(1,1)-P(1,3)}{(z_3-z_1)P(1,3)}=\frac{a_1-G(z_3)}{(z_3-z_1)G(z_3)}  
\end{equation}
Next we express $P(3,3)$ with recursion relation
\begin{equation}
a_3 = P(3,3)= \frac{P(2,2)-P(2,3)}{(z_3-z_2)P(2,3)}=\frac{a_2-\frac{a_1-G(z_3)}{(z_3-z_1)G(z_3)} }{(z_3-z_2)\frac{a_1-G(z_3)}{(z_3-z_1)G(z_3)} }
\end{equation}
which is equivalent to
\begin{equation}
1+a_3(z_3-z_2) = \frac{a_2}{\frac{a_1-G(z_3)}{(z_3-z_1)G(z_3)} }
\end{equation}
or
\begin{equation}
\frac{a_2(z_3-z_1)}{1+a_3(z_3-z_2)} = \frac{a_1}{G(z_3)}-1 
\end{equation}
or
\begin{equation}
\frac{a_1}{1+\frac{a_2(z_3-z_1)}{1+a_3(z_3-z_2)}} = G(z_3)
\end{equation}
hence $P(3,3)$ is indeed $a_3$.

\end{widetext}

\end{document}
